\clearpage
\section*{Instruções}
Cada questão possui: 
  \begin{itemize}
  \item instruções de implementação, 
  \item um prótótipo da função a ser implementada e 
  \item uma função de teste.
  \end{itemize}

                      
Seja o seguinte exemplo:
\subsection*{Questão Exemplo}
\begin{verbatim}
Objetivo: Somar dois números inteiros.
Entrada : Dois números inteiros.
Saída   : A soma dos números.
\end{verbatim}

\begin{lstlisting}
 int soma (int num1, num2);
 void  test()
 {
    puts(soma(6, 4) == 10);
    puts(soma(3,-4) == -1);
 }
\end{lstlisting}

Você deve implementar a função soma e executar o código de teste. O ambiente completo do arquivo main.c seria
algo parecido com o exemplo a seguir.

\begin{lstlisting}
#include <stdio.h>

 int soma (int num1, num2)
 {
      return num1 + num2; //essa aqui é a sua implementacao
 }
 
 void  test()
 {
    puts(soma(6, 4) == 10);
    puts(soma(3,-4) == -1);
 }
 
 int main()
 {
    test();//aqui voce chama o codigo de teste.
    return 1;
 }
\end{lstlisting}

Se sua implementação da função estiver correta, como aconteceu na função soma, será impresso o número 1 na tela. Caso contrário, 
será impresso o número 0. Assim, fica mais fácil de acompanhar em quais casos seu código está falhando.
